\section{Conclusions} \label{sec: conc}
Certain points of interest can be highlighted after the study of the system. 
Firstly, it is important to note the simplicity of the method used in this study. It is truly remarkable that basic algebra and the principle of energy conservation are the only tools necessary to obtain the response of complex systems. This relatively simple example perfectly demonstrates the power of the SEA method for analyzing more complex structures. However, it is important to remember that in this case, the characterization of the different subsystems (obtaining modal densities and CLFs) was given in the problem statement, and the real complexity of this method may come when calculating these values for more complex subsystems.

In conclusion, this project has clarified the concepts taught in the second part of the course regarding vibroacoustics of high-frequency systems.