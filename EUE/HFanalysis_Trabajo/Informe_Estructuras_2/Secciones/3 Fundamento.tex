\section{Theoretical foundation} \label{sec: ft}

As previously stated in Section \ref{sec: stat}, the purpose of this study is to obtain the model's response to variable excitation in the high-frequency range. To achieve this, the Statistical Energy Analysis method is employed, which is elucidated in this section.


\subsection{Statistic energy analysis}
Statistical Energy Analysis (SEA) is a method used to analyze the vibrational behavior of complex structures, especially those subjected to high-frequency excitations. The method is based on statistical techniques and assumes that the vibrational energy of a system can be partitioned into discrete energy "bins" that are statistically independent.

To apply the SEA approach, a complex structure is divided into interconnected subsystems, each with its own set of vibrational modes. The energy stored in each subsystem is modeled as a statistical energy distribution, where each mode of vibration behaves as an independent, damped harmonic oscillator. The energy exchange between subsystems is described by coupling coefficients, known as coupling loss factors.

Given the systems $i$ and $j$ (with $i\neq j$) the conservation of the power flux can be applied as,

\begin{align}
P_{i,in} &= P_{i,diss} + P_{ij}\\
P_{j,in} &= P_{j,diss} + P_{ji}
\end{align}

where $P_{i,diss}$ is the internal dissipated power, defined as 
\begin{equation}
P_{i,diss} = \eta_i \omega E_i
\end{equation}
and $P_{ij}$ is the energy exchanged, 
\begin{equation}
P_{ij} = \eta_{ij} \omega E_i -\eta_{ji} \omega E_j
\end{equation}

The combination of these equations result in 

\begin{equation}
P_{i,in} = \eta_i \omega E_i + \omega \sum_{i=0}^N  (\eta_{ij}  E_i -\eta_{ji}  E_j),
\end{equation}

then, knowing that the modal density is 

\begin{equation}
n_i = \frac{N_i}{\Delta \omega},
\end{equation}

there is one last equation derived from the reciprocity relationship between the modal densities:


\begin{equation}
\eta_{ij} n_i = \eta_{ji} n_j
\label{eq: reciproc}
\end{equation}

Applying this process to the analysis system yields the results presented in \autoref{sec: res}.




